\documentclass[class=article,crop=false]{standalone}

\usepackage[subpreambles=true]{standalone}
\usepackage{import}

\usepackage{amsmath}
\usepackage{multicol}
\usepackage{setspace}

\newenvironment{bcases}{%
    \left[
    \renewcommand{\arraystretch}{1.2}%
    \array{@{}l@{\quad}l@{}}
    }
    {\endarray\right.}

\onehalfspacing
\begin{document}
    \section*{Задание 1}
    Вычислить значения HITS (<<hubs and authorities>>) для данного графа.

    \begin{figure*}[h]
        \centering
        \subimport{./}{graph.tex}
    \end{figure*}

    \begin{multicols}{2}
        \noindent Рассмотрим матрицу смежности графа $\mathbf{M}$ и~транспонированную $\mathbf{M^T}$:
        \begin{gather*}
            \mathrm{M}=
            \begin{pmatrix}
                1 & 1 & 1 \\
                1 & 0 & 1 \\
                0 & 1 & 1
            \end{pmatrix},
            \quad
            \mathrm{M^T}=
            \begin{pmatrix}
                1 & 1 & 0 \\
                1 & 0 & 1 \\
                1 & 1 & 1
            \end{pmatrix}
        \end{gather*}

        \noindent\textbf{Шаг~0.} Зададим начальные значения векторов <<каталожности>> (<<hubbiness>>) $\mathbf{h}$ и <<авторитетности>> (<<authority>>) $\mathbf{a}$ таким образом, чтобы $\mathrm{h}_0\left( i\right) =\mathrm{a}_0\left( i\right) =1$ для любой вершины $i$.

        \noindent\textbf{Шаг~1.} Вычислим новые значения $\mathbf{h}$ и~$\mathbf{a}$:
        \begin{gather*}
            \begin{cases}
                \mathrm{h}_k=\phi_k\mathrm{M}\mathrm{a}_{k-1}\\
                \mathrm{a}_k=\psi_k\mathrm{M^T}\mathrm{h}_{k-1}
            \end{cases}
            \rightarrow\quad
            \begin{cases}
                \mathrm{h}_k=\phi_k\psi_{k-1}\mathrm{M M^T}\mathrm{h}_{k-2}\\
                \mathrm{a}_k=\phi_{k-1}\psi_k\mathrm{M^T M}\mathrm{a}_{k-2}
            \end{cases}
        \end{gather*}
        где $\phi_k$ и~$\psi_k$ --- такие числа, что для любого $k>1$ верно
        $\left|\phi_k\mathrm{M}\mathrm{a}_{k-1}\right| =\left|\psi_k\mathrm{M^T}\mathrm{h}_{k-1}\right| =1$.

        Мы видим, что $\mathbf{h}$ и $\mathbf{a}$ --- собственные векторы матриц $\mathbf{M M^T}$ и $\mathbf{M^T M}$ соответственно.

        \noindent\textbf{Шаг~2.} Рассмотрим $\mathbf{M M^T}$ и~$\mathbf{M^T M}$:
        \begin{gather*}
            \mathrm{M M^T}=
            \begin{pmatrix}
                3 & 2 & 2 \\
                2 & 2 & 1 \\
                2 & 1 & 2
            \end{pmatrix},
            \quad
            \mathrm{M^T M}=
            \begin{pmatrix}
                2 & 1 & 2 \\
                1 & 2 & 2 \\
                2 & 2 & 3
            \end{pmatrix}
        \end{gather*}

        \noindent Полученные матрицы симметричны и положительно определены, следовательно их собственные значения положительны. Значит векторы $\mathbf{h}$ и~$\mathbf{a}$ сходятся к собственным векторам матриц  $\mathbf{M M^T}$ и $\mathbf{M^T M}$, соответсвующим их наибольшим собственным значениям.
        \vfill\null\columnbreak

        Найдём эти собственные векторы:
        \begin{align*}
            \left|\mathrm{M M^T}-\lambda\mathrm{E}\right| &=
            \begin{vmatrix}
                3-\lambda & 2         & 2         \\
                2         & 2-\lambda & 1         \\
                2         & 1         & 2-\lambda
            \end{vmatrix}=0,\\
            \left|\mathrm{M^T M}-\mu\mathrm{E}\right| &=
            \begin{vmatrix}
                2-\mu & 1     & 2     \\
                1     & 2-\mu & 2     \\
                2     & 2     & 3-\mu
            \end{vmatrix}=0
        \end{align*}

        \noindent Не трудно заметить, что решения обоих уравнений получатся одинаковые:
        \begin{gather*}
            \begin{bcases}
                \lambda_1=\mu_1=1\\
                \lambda_2=\mu_2=3-2\sqrt{2}\\
                \lambda_3=\mu_3=3+2\sqrt{2}
            \end{bcases}
        \end{gather*}

        \noindent\textbf{Шаг~3.1.} Решим уравнение $\left(\mathbf{M M^T}-\lambda\mathbf{E}\right)\mathbf{h}=0$, подставив $\lambda=\lambda_3=3+2\sqrt{2}$:
        \begin{gather*}
            \begin{pmatrix}
                -2\sqrt{2} & 2            & 2            \\
                2          & -1-2\sqrt{2} & 1            \\
                2          & 1            & -1-2\sqrt{2}
            \end{pmatrix}\mathrm{h}=
            \begin{pmatrix}
                0 \\
                0 \\
                0
            \end{pmatrix}
            \longrightarrow\\
            \mathrm{h}=
            \begin{pmatrix}
                \sqrt{2}/2 \\
                1/2        \\
                1/2
            \end{pmatrix}
        \end{gather*}

        \noindent\textbf{Шаг~3.2.} Решим уравнение $\left(\mathbf{M^T M}-\mu\mathbf{E}\right)\mathbf{a}=0$, подставив $\mu=\mu_3=3+2\sqrt{2}$:
        \begin{gather*}
            \begin{pmatrix}
                -1-2\sqrt{2} & 1            & 2          \\
                1            & -1-2\sqrt{2} & 2          \\
                2            & 2            & -2\sqrt{2}
            \end{pmatrix}\mathrm{a}=
            \begin{pmatrix}
                0 \\
                0 \\
                0
            \end{pmatrix}
            \longrightarrow\\
            \mathrm{a}={
                \begin{pmatrix}
                    1/2 \\
                    1/2 \\
                    \sqrt{2}/2
                \end{pmatrix}
            }
        \end{gather*}
    \end{multicols}
\end{document}