\documentclass[class=article,crop=false]{standalone}

\usepackage[subpreambles=true]{standalone}
\usepackage{import}

\usepackage{amsmath}
\usepackage{setspace}

\newenvironment{bcases}{%
    \left[
    \renewcommand{\arraystretch}{1.2}%
    \array{@{}l@{\quad}l@{}}
    }
    {\endarray\right.}

\onehalfspacing
\begin{document}
    \section*{Задание 2}
    Вычислить значения HITS для данного графа с $m$ вершинами.

    \begin{figure*}[h]
        \centering
        \subimport{./}{graph.tex}
    \end{figure*}

    \noindent Рассмотрим матрицу смежности графа $\mathbf{M}$ ($m\times m$) и~транспонированную $\mathbf{M^T}$:
    \begin{gather*}
        \mathrm{M}=
        \begin{pmatrix}
            1      & 1      & 0      & \cdots & 0      \\
            0      & 0      & 1      & \cdots & 0      \\
            \vdots & \vdots & \vdots & \ddots & \vdots \\
            0      & 0      & 0      & \cdots & 1      \\
            0      & 0      & 0      & \cdots & 0
        \end{pmatrix},\quad
        \mathrm{M^T}=
        \begin{pmatrix}
            1      & 0      & \cdots & 0      & 0      \\
            1      & 0      & \cdots & 0      & 0      \\
            0      & 1      & \cdots & 0      & 0      \\
            \vdots & \vdots & \ddots & \vdots & \vdots \\
            0      & 0      & \cdots & 0      & 0
        \end{pmatrix}
    \end{gather*}

    \noindent Теперь рассмотрим произведения $\mathbf{M M^T}$ и~$\mathbf{M^T M}$:
    \begin{gather*}
        \mathrm{M M^T}=
        \begin{pmatrix}
            2      & 0      & \cdots & 0      & 0      \\
            0      & 1      & \cdots & 0      & 0      \\
            \vdots & \vdots & \ddots & \vdots & \vdots \\
            0      & 0      & \cdots & 1      & 0      \\
            0      & 0      & \cdots & 0      & 0
        \end{pmatrix},\quad
        \mathrm{M^T M}=
        \begin{pmatrix}
            1      & 1      & 0      & \cdots & 0      \\
            1      & 1      & 0      & \cdots & 0      \\
            0      & 0      & 1      & \cdots & 0      \\
            \vdots & \vdots & \vdots & \ddots & \vdots \\
            0      & 0      & 0      & \cdots & 1
        \end{pmatrix}
    \end{gather*}

    \noindent Пользуясь рассмотренным при решении предыдущего задания алгоритмом получаем ответ:
    \begin{gather*}
        \mathrm{h}=
        \begin{pmatrix}
            1      \\
            0      \\
            0      \\
            \vdots \\
            0
        \end{pmatrix},\quad
        \mathrm{a}=
        \begin{pmatrix}
            \sqrt{2}/2 \\
            \sqrt{2}/2 \\
            0          \\
            \vdots     \\
            0
        \end{pmatrix}
    \end{gather*}
\end{document}